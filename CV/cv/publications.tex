%-------------------------------------------------------------------------------
%	SECTION TITLE
%-------------------------------------------------------------------------------
\cvsection{Publications}


\cvsubsection{Peer Reviewed}

%-------------------------------------------------------------------------------
%	CONTENT
%-------------------------------------------------------------------------------\
\cvparagraph
\begin{itemize}
\item Capriotti, J., and Y. Li, 2022, Clustering inversion of electrical potential due to an arbitrarily anisotropic layered half-space. Journal of Geophysical Research: Solid Earth, \textit{in review}.
\item Capriotti, J., and Y. Li, 2022, Fluid-flow coupled time-lapse gravity inversion for permeability and porosity distributions, Geophysical Journal International \textit{in review}.
\item Capriotti, J., and Y. Li, 2022, Joint inversion of gravity and gravity gradient data: a systematic evaluation. Geophysics, 87:2, G29-G44.
\item Capriotti, J., and Y. Li, 2015, Inversion for permeability distribution from time-lapse gravity data. Geophysics, 80, WA69–WA83.
\end{itemize}
\clearpage
\cvsubsection{In Proceedings}
\begin{itemize}
\item Capriotti J., J. Kuttai, D. Fournier, Heagy L.J., Linking open source tools for geophysical simulation and inversion in rugged topographies, AGU Fall Meeting 2022.
\item Heagy L.J., T. Astic, J. Capriotti, D.W. Oldenburg, Carbon Sequestration in ultramafic rocks and the role of geophysical inversions. AGU Fall Meeting 2021.
\item Capriotti. J., T. Astic, L.J. Heagy, D.W. Oldenburg, Implementing an open-source framework to joint inversion. AGU Fall Meeting 2021.
\item Capriotti. J., L.J. Heagy, J. Kuttai, Geophysical Simulations and Inversions with SimPEG. Engineering and Mining Geophysics 2021. p.1 - 2
\item Capriotti. J., S. Kang, D. Cowan, L.J. Heagy, D.W. Oldenburg, Open-source direct current resistivity software development for groundwater applications. AGU Fall Meeting Abstracts, 2020.
\item Capriotti, J., and Y. Li, 2019, Joint equivalent source processing of gravity and gravity gradient data. International Workshop and Gravity, Electrical \& Magnetic Methods and their Applications, Xi'an, China, 19-22 May 2019: pp. 324-327.
\item Capriotti, J., and Y. Li, 2017, Geomodelling with Minecraft: Geophysics meets video games. In “Proceedings of Exploration 17: Sixth Decennial International Conference on Mineral Exploration” edited by V. Tschirhart and M.D. Thomas, pp. 729–733
\item Capriotti, J., Y. Li, and R. Krahenbuhl, 2015, Joint inversion of gravity and gravity gradient data using a binary formulation. International Workshop and Gravity, Electrical \& Magnetic Methods and their Applications, Chenghu, China, 19-22 April 2015: pp. 326-329.
\item Kass, M. Andy, B. J. Drenth, L. Foks, and J. Capriotti, 2015, Quantitative geophysical interpretation of gravity gradient and magnetic data over a buried carbonatite: The Elk Creek deposit, Nebraska, USA. International Workshop and Gravity, Electrical \& Magnetic Methods and their Applications, Chengdhu, China, 19-22 April 2015: pp. 201-204.
\end{itemize}
\cvsubsection{Expanded Abstracts}
\begin{itemize}
\item Capriotti J., L.J. Heagy, D. Cowan, J. Kuttai, S. Kang, D. Fournier, T. Astic, R. Cockett, and D.W. Oldenburg, 10 years of SimPEG: Recent developments and the next steps forward, KEGS 2023.
\item Kang, S., J. Capriotti, D.W. Oldenburg, L.J. Heagy, and D. Cowan, Open-source geophysical software development for groundwater applications. SEG Technical Program Expanded Abstracts 2020: pp. 1989-1993.
\item Oldenburg, D.W., L.J. Heagy, S. Kang, D. Cowan, J. Capriotti, K. Fan, M. Maxwell, The role of open source resources and practices in capacity building. SEG Technical Program Expanded Abstracts 2020: pp. 3361-3365.
\item Fan, K., D.W. Oldenburg, M.Maxwell, D. Cowan, S. Kang, L.J. Heagy, J. Capriotti, Improving water security in Mon State, Myanmar via geophysical capacity building. SEG Technical Program Expanded Abstracts 2020: pp. 3355-3360.
\item Capriotti, J. and Y. Li, 2019, Equivalent source processing of vector gravity data. SEG Technical Program Expanded Abstracts 2019: pp. 1764-1768.
\item Capriotti, J. and Y. Li, 2018, Guided fuzzy c-means clustering inversion of electrical potential due to an anisotropic-layered half-space. SEG Technical Program Expanded Abstracts 2018: pp. 914-918.
\item Capriotti, J. and Y. Li, 2017, Time-lapse gravity inversion for multiple reservoir parameters using fuzzy C-means clustering. SEG Technical Program Expanded Abstracts 2017: pp. 5865-5869.
\item Maag, E., J. Capriotti, and Y. Li, 2017, 3D gravity inversion using the finite element method. SEG Technical Program Expanded Abstracts 2017: pp. 1713-1717.
\item Capriotti, J. and Y. Li, 2015, Integrating Gravity and Gravity Gradiometry Data for Joint Inversion: A Case Study at the Kauring Test Site. SEG Technical Program Expanded Abstracts 2015: pp. 1505-1509.
\item Capriotti, J. and Y. Li, 2014, Gravity and gravity gradient data: Understanding their information content through joint inversions. SEG Technical Program Expanded Abstracts 2014: pp. 1329-1333.
\item Capriotti, J. and Y. Li, 2013, Aquifer storage monitoring at Leyden Mine using time-lapse gravity: A revisit seven years later. SEG Technical Program Expanded Abstracts 2013: pp. 1146-1150.
\end{itemize}

